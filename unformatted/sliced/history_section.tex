\documentclass{report}[11pt]
\usepackage{graphicx}
\usepackage{float}
\usepackage{amsmath}
\usepackage{amssymb}
\usepackage{gensymb}
\usepackage{times}
\usepackage{setspace}
\usepackage{fullpage}
\usepackage{cancel}
\usepackage{listings}
%\usepackage{breqn}
\usepackage{wrapfig}
%\usepackage{feynmp}
%\usepackage{feynmp-auto}
%\usepackage{slashed}
\usepackage{hyperref}

\setcounter{tocdepth}{2}

\newcommand{\sutwo}[4]{
\left(
\begin{array}{cc}
#1 & #2 \\
#3 & #4 \\
\end{array}
\right)
}

\newcommand{\fovec}[4]{
\left(
\begin{array}{c}
#1\\
#2\\
#3\\
#4\\
\end{array}
\right)
}

\newcommand{\trevec}[3]{
\left(
\begin{array}{c}
#1\\
#2\\
#3\\
\end{array}
\right)
}
 
\newcommand{\spinor}[2]{
\left(
\begin{array}{c}
#1\\
#2\\
\end{array}
\right)
}

\newcommand{\spin}[2]{
\left(
\begin{array}{c}
#1\\
#2\\
\end{array}
\right)
}

\newcommand{\spind}[2]{
\left(
#1, #2
\right)
}
%\newcounter{modelsec}[subsection]
%\renewcommand{\themodelsec}{\Alph{modelsec}}
%\newcommand{\modelsec}[1]{
 % \refstepcounter{modelsec}
 % \subsubsection*{\centering\themodelsec. #1}
%}
\newcommand{\begeq}[1]{
\begin{equation}
#1
\end{equation}}

\newcommand{\parfrac}[2]{
\frac{\partial #1}{\partial #2}
}
\newcommand{\diffrac}[2]{
\frac{d #1}{d #2}
}

%\newcommand{\feyndiag}[2]{
%\frac{d #1}{d #2}
%}

\newcommand{\thermoderv}[2]{
\left(\frac{\partial#1}{\partial#2}\right)
}
\newcommand{\constderv}[3]{
\left(\frac{\partial #1}{\partial #2}\right)_{#3}
}
\newcommand{\partwo}[2]{
\frac{\partial^2 #1}{\partial #2 ^2}
}
\newcommand{\begsp}[1]{
\begin{equation}
\begin{split}
#1
\end{split}
\end{equation}}


\newcommand{\figwrap}[4]{ %1 is caption 2 is filename 3 is width 4 is l, c or r for justification
\begin{wrapfigure}{#4}{#3}
\includegraphics[width=#3]{#2}

\caption{#1\label{#2}}
\end{wrapfigure} }


\newcommand{\figin}[3]{
\begin{figure}[H]
\begin{center}
\vspace{-10pt}


    \includegraphics[width= #3]{./#2.png}
    \\

  \caption{#1} 
        \label{#2}
\end{center}
\end{figure}}

\newcommand{\figindos}[5]{
\begin{figure}[H]
\begin{center}
\vspace{-10pt}


    \includegraphics[width= #3]{./#2.png}
    \includegraphics[width= #5]{./#4.png}
    \\

  \caption{#1} 
        \label{#2}
\end{center}
\end{figure}}

\renewcommand\thesection{\arabic{section}}
\renewcommand\thesubsection{\thesection.\alph{subsection}}
%\renewcommand{\thesubsection}{\thesection.\roman{subsection}}

\title{Electronics and Timing for the AugerPrime Upgrade and Analysis of Starburst Galaxies as Sources of Ultra High Energy Cosmic Rays}
\author{Rob Halliday}
\date{}

\numberwithin{section}{chapter}

\begin{document}
%\large
\unitlength=2pt
\maketitle
\doublespace
\chapter{Introduction}
In the development of modern physics throughout the last century, the science of cosmic rays, particles originating outside of Earth's atmosphere, have been a driving force. 
Since their first observation by Coulomb and subsequent exploration by Domenico Pacini and separately Victor Hess, the study of particles accelerated by extraterrestrial mechanisms has been a main source of knowledge, leading to the founding of particle physics, astrophysics, and more recently particle astronomy \cite{pacini}.  Cosmic rays are known to originate from a variety of sources, in particular low and medium energy cosmic rays are believed to originate from supernovae and their remnants, while the highest energy cosmic rays are still of unknown origin \cite{stanev_rev}. 
% The lowest energy particles originate from the sun and interactions with solar radiation in the Earth's atmosphere. 

The study of Ultra High Energy Cosmic Rays (UHECRs) originated with experiments by Pierre Auger and his colleagues to detect extensive air showers, and was solidified by the work of John Linsley and collaborators in the experiments at Volcano Ranch \cite{linsley}. The developments in the detection mechanisms and source identification that will be discussed in this document, follow along the line of experimentation propagated by Linsley, using large arrays of detectors spread over vast distances to detect extensive air showers. Throughout the latter half of the 20th century, various techniques were used to detect UHECRs including films and electronic arrays of scintillators and water tanks. These efforts culminated in the construction of the Pierre Auger Observatory. 

Initially built as an array of 1600 Water Cherenkov detectors and 48 Nitrogen Fluorescence telescopes, covering 3600 km$^2$, approximately the size of the American state of Rhode Island, the Pierre Auger Observatory has been the flagship detector for particles above a kinetic energy of $10^{18}$ eV. As time progressed, the observatory integrated many efforts to improve its ability to characterize and detect extensive air showers \cite{enhancements}. After about a decade of operation, the possibility of upgrading the observatories capabilities across all detectors was explored, and now we are on the threshold of executing an array-wide improvement in the electronics and detection abilities of the apparatus \cite{firstprime}.  

\section*{A brief history of Comic Rays and Particle Physics}
In the post World War One era of physics, many efforts were made to understand and utilize the advances of Quantum Mechanics and General and Special Relativity. In 1926, Erwin Schr�dinger posited his eponymous equation, describing the dynamics of wavefunctions, the fundamental description of quantum particles.  Shortly thereafter, in 1928, Paul A. M. Dirac proposed his equation which united the principles of Quantum Mechanics and Special Relativity \cite{dirac}. It is with this advance, that the importance of cosmic rays takes center stage.  

Integral to the Dirac equation, is the prediction of opposite-charge equal-mass particles for each known elementary particle. At the time, knowledge of elementary particles was extremely limited and this led only to the prediction of the positron. In 1933, Carl Anderson published his paper confirming the existence of a "positive electron" which was quickly followed, just 3 years later, by the discovery of the muon \cite{positron,muon}. In the years after this, cloud chambers with scintillator triggered cameras led to the discovery of many new particles, most of which were types of mesons.  
\figwrap{The photo published by Carl Anderson in \cite{positron} in 1932, which led to the confirmation of the positron's existence. The cosmic rays entering his cloud chamber were put under a magnetic field, hence the bending trajectory, and pictured in the center is a lead plate, meant to stop electrons.}{./images/positron.png}{2 in}{l}
Meanwhile, Pierre Auger and his collaborators were working to uncover particles of extremely high initial energies causing showers of lower energy particles spread over great distances \cite{firstshowers}. He and his team did this by setting up scintillation counters at increasing separations from each other and watching the rate at which coincidences were detected as a function of their separation. They tried this at multiple sites while monitoring external conditions such as barometric pressure. With this experiment, the existence of particles estimated to have energies of upwards of 10$^16$ eV was confirmed. While many of the advances in particle physics at this time were driven by lower energy cosmic rays, this represents the first venture towards detecting Ultra High Energy Cosmic Rays. 

After the second world war, the technological and science funding situations conspired to provide a previously-unseen level of funding for experimentation (particularly into the mid and late '50s) \cite{scifund1}. Additionally, the advent of the photomultiplier tube (PMT) and the application of the first computers gave cosmic ray and particle physicists new tools \cite{pmthistory}. While versions of the photomultiplier tube existed in the 1930's, the products available in the late 40's and 50's had been refined from their predecessors. The photomultiplier tube is a device consisting of a photocathode-- a thin piece of metal off of which free electrons are generated from photons through the photoelectric effect-- multiple metal leaves and an anode to collect freed charge. A large voltage is held between the anode and the cathode such that when an electron comes off of the cathode, it is pulled towards the anode. It excites many more electrons off of the metal leaves and these electrons are similarly accelerated towards the anode by the high voltage. Through this process, a photomultiplier tube takes a photon and produces a measurable current on the order of nano or micro-amperes. 

During this time, we begin to see a split between particle physics and cosmic ray physics. While particle accelerators had existed since the 20's (or earlier depending on how rigorously `particle acceleration' is defined), they were now becoming much more useful as fundamental probes of physics in ways that cosmic rays simply could not match up to \cite{colliderhistory1}. Fundamentally, the difference between studying particle physics using a collider versus using cosmic rays is in the predictability of the point and time of interaction. Cosmic rays, even today, provide access to much higher energy interactions, but they occur in-effect randomly, while interactions in a collider occur in a predetermined location and time (some like to say ``in a jar"). Due to this predictability, particle physicists need to only build one detection apparatus and guarantee the interesting portions of the collision will occur inside this apparatus. Meanwhile, cosmic ray physics at this time, especially at the highest energies, began to be engineered in a distributed way, opting for dispersed detectors in the vein of Pierre Auger's experiments in the '30s. 
\figwrap{A diagram of Volcano Ranch in 1962 (\cite{volranch}): the dark circle represents a new detector being added for in depth shower timing studies.}{./images/volranch.png}{2 in}{r} 
Perhaps the most notable experiment\footnote{There were many influential and noteworthy experiments that shaped the field, but here I will concentrate on an outline of experiments to show how the field of particle astrophysics advanced into the creation of the Auger Observatory.} pursuing this direction was the Volcano Ranch Experiment, led by John Linsley \cite{linsley}. Linsley greatly benefitted from the experimental agility of living in a time when a ``large'' collaboration was on the order of dozens of scientists (a trend started by Case Western's own Albert Michelson). His work at Volcano Ranch began with the setup of an array of scintillation counters, commonly referred to as scintillators (although this is a slight misnomer, see section \ref{scints}). These scintillators were placed at uniform distances apart from each other in a triangular grid. The grid spacing changed as the experiment progressed, but to give context, in 1962 it was 884m \cite{volranch}. Due to the perhaps unimpressive speed of the electronics of his day, it was still debated whether the arrival times of the signals from each scintillation detector could be used to triangulate the arrival directions of the showers, thus increasing the grid spacing gave longer differences in arrival times and therefore made timing reconstruction a more viable approach. That said, for many of the great discoveries of Volcano Ranch, the coincidence window for detection was set such that only showers of about 10\degree zenith would be recorded. 

It was in this configuration that Linsley's group detected the first $10^20$eV comic ray. Along with a handful of other data points in this regime, Linsley was able to see the flattening of the spectrum \cite{linspec}, as will be discussed in the coming sections of this chapter. The flattening and steepening of the effective radius of curvature of cosmic ray showers in various experiments was also noted at this time, and it was correctly assumed by Linsley and others in the field that this relates to the point of first interaction in the atmosphere.

As is often seen in science and technological development, great ideas are frequently had in multiple places independently around the same time. This is the case of the Air Fluorescence camera/technique. At Volcano Ranch by Linsley's group, as well as the Sydney Air Shower Array detectors called Fluorescence Telescopes (see \autoref{fluor}) were implemented. The design comes from the group of Kenneth Greissen, a prolific graduate student of prolific cosmic ray researcher Bruno Rossi, who was working on it through the mid 1960's but was somewhat unsuccessful (although his work did result in the construction of the Fly's Eye detectors) \cite{ultraray}. The fluorescence telescope is one of the inventions that led to the deep understanding of cosmic ray showers as one could now visualize how the shower develops as it goes through the atmosphere. This allowed confirmation and further study of important quantities such as $X_{max}$ and the elongation rate. 

Before moving on (or back, as the case may be) to emulsion film detectors, two honorable and important mentions must be given. First, the Haverah park experiment of the University of Leeds in England, including Alan Watson, a great proponent and leader of UHECR research, became one of the top High Energy Cosmic Ray detectors around the same time as the deployment of Volcano Ranch (although it only truly came into full operation about 5 years after both experiments were commissioned). The array initially consisted of 4 water Cherenkov detectors in a semi-triangular grid (as triangular as it can be with only 4 detectors) and was triggered in much the same way as the Volcano Ranch Experiment \cite{haverah_lillicrap}. While Volcano Ranch had a much larger area than Haverah Park at the time, more advanced electronics, and upgrades over the years allowed the experiment to produce data that was still being usefully analyzed well into the 1990's \cite{haverah_watson}. In many ways, Haverah Park advanced the work of previous water Cherenkov experiments, which had problems with fungus and water purity \cite{haverah_who}. A main breakthrough from the water Cherenkov technique was a more favorable cost vs. aperture curve over scintillators (see \autoref{wcd}). 

In a competing lineage to electronic detectors, emulsion film detectors were developed and deployed in the late 1930's by Marietta Blau, who deployed them high into the Alps \cite{ultraray_blau}. These films acted much like a film picture from a camera, except that they are exposed by either high energy ionizing radiation or X-rays (as in the analogous medical application). These films allowed cosmic ray researchers from the late thirties well into the eighties to see the types of tracks left by showers over long exposure times \cite{crapp}. While these measurements are not inherently calorimetric, the precision given by being able to visually see the tracks of ionizing particles allows for particle counting and shower geometry methods to attempt to determine the primary energy. These types of experiments allowed researchers to directly visualize the nature and progression of cosmic ray showers, allowing them to confirm and give insight into the results of other experiments which were better at collecting larger numbers of showers at higher energies, but were all electronically operated and could not give intuitive insight. 

Looking forward from the early days of Volcano Ranch and Haverah Park, the late 1960's through the 80's opened new doors for cosmic rays physics, especially in terms of new ``messenger'' particles being looked at. In this vein, new detection techniques were provided by both the Whipple Telescope, the first Imaging Atmospheric Cherenkov Telescope (IACT) and the KamiokaNDE experiment, a proton decay experiment that accidentally became the first effective neutrino telescope. The Whipple telescope is based on a Davies-Cotton optics design from the mid 1950's which uses segmented mirrors to create a large aperture that is in turn used to collect the direct Cherenkov light from air showers caused by GeV to TeV gamma rays. This proved the viability of what are now frequently referred to as gamma ray telescopes and almost all such designs are derived from the Whipple design. These include VERITAS (at the same site as the Whipple Telescope), MAGIC and HESS, as well as the upcoming CTA project.  

On the neutrino front, the KamiokaNDE I and II experiments ran through the 80's, beginning in 1983, and while they intended to detect proton decay, their detectors were sensitive enough that they were able to detect atmospheric, solar and astrophysical neutrinos. The apparatus is a 3000 metric ton pool of water, which is ideal for a proton decay experiment as it contains a high concentration of hydrogen atoms and facilitates Cherenkov light production \cite{kamiokande}. That said, the large fiducial volume of the detector, the sides of which are lined with 1000 PMTs specially designed by Hammamtsu, makes it such that any high energy particle interacting inside will be caught (as it must for their background rejection). Famously, this experiment caught neutrinos from Supernova 1987A, which were likely the first ever astrophysical neutrinos measured and identified. 

In the transition from the late 1980's to the early 1990's, the field of particle astrophysics, especially UHECR astrophysics, began to resemble its current form. Through the interest of Nobel laureate James W Cronin, and other particle physicists, as well as X-ray astronomers, the goal of figuring out the acceleration mechanisms of UHECRs came to the front of the field. Accordingly, a number of sensitive and precisely executed experiments were built; for the sake of brevity, I will highlight two: the Akeno Giant Air Shower Array (1990-2004) and CASA-MIA (1990-1997). The Akeno Giant Air Shower Array (AGASA) was a 100 km$^2$ array in Japan at the Akeno Observatory consisting of surface and buried scintillation detectors and was built with the explicit intention of observing cosmic rays of $10^17$ eV or higher. Some infrastructure and detectors were already installed as early as 1984 from other ventures \cite{agasa}. AGASA paved the way for the Auger Observatory, but was the direct predecessor, along with Fly's Eye, to the Telescope Array \cite{ultraray}. 

CASA-MIA, or the Chicago Air Shower Array-Michigan Muon Array, set out with a slightly different objective. Instead of looking for the showers caused by hadronic primaries, the objective of CASA-MIA was to explore the arrival directions of Very High Energy (VHE) gamma rays. In the optical to low gamma ray energy regimes, only direct detection of photons is possible, usually by balloon- or space-borne detectors, but gamma rays above ~ 10 GeV cause air showers much like UHECRs except scaled down and of much lower muon content and higher $X_{max}$. To detect these, about one thousand surface stations consisting of 4 scintillation counters (CASA) each, topped with an ancient lead (i.e. low radioactivity) sheet to encourage pair production in shower photons, and about one thousand more buried scintillation counters (MIA) were employed in a .23 km$^2$ array \cite{casamia}. 

 The basic idea is to catch the showers from \~ 10$^{14}$ eV - 10$^{16}$ eV gamma rays, and reject hadronic showers of similar energies through their muon multiplicity. Perhaps ironically, the CASA-MIA team provided much of the expertise and drive for the construction of Auger, but the detector itself is more of a functional predecessor to the High Altitude Water Cherenkov Experiment (HAWC), a gamma ray observatory which uses air shower techniques and direct particle detection. Through the CASA-MIA experiment, much was learned about how to set up a distributed array and how to design effective detection logic. 

This more or less brings us to the beginning of the current era of UHECR physics, which is dominated by the Auger Collaboration, of which I am a member, and the Telescope Array (TA). A more detailed description of both experiments will be given in \autoref{auger} and \autoref{ta}, respectively, but to give context, the Pierre Auger Observatory started development in the late 1990's and began construction in 2000 in Malargue, Mendoza Province, Argentina, at the turn of the millennium. Operation began in 2004, and has continued through to today. The TA experiment began development around the same time as Auger began operation, and is located at Dugway Proving grounds, in a site expanded around the Fly's Eye experiment. Both apparatuses measure UHECRs of 10$^{18}$ eV+ throughout the bulk of the array, and feature low energy infills for engineering tests and to expand their sensitivities to energies as low as 10$^{16}$ eV.








 


\begin{thebibliography}{9}
\bibitem{pauger}
L Persson (1996) Pierre Auger-A Life in the Service of Science, Acta Oncologica, 35:7, 785-787, DOI: \url{https://doi.org/10.3109/02841869609104027}

\bibitem{stanev_rev}
A Letessier-Selvon and T Stanev (2011) Ultra High Energy Cosmic Rays, Rev. Mod. Phys. 83, 907, DOI: \url{https://doi.org/10.1103/RevModPhys.83.907}

\bibitem{pacini}
N Giglietto (2011) The contribution by Domenico Pacini to the Cosmic Ray Physics, DOI: 10.1016/j.nuclphysbps.2011.03.002, \url{https://arxiv.org/abs/1101.0398v1}

\bibitem{linsley}
J Linsley (1963) Evidence for a Primary Cosmic-Ray Particle with Energy $10^{20}$ eV, \url{https://doi.org/10.1103/PhysRevLett.10.146}

\bibitem{volranch}
J Linsley, L Scarsi (1962) Arrival Times of Air Shower Particles at Large Distances from the Axis, \url{https://doi.org/10.1103/PhysRev.128.2384}

\bibitem{linspec}
J Linsley, L Scarsi (1962) Arrival Times of Air Shower Particles at Large Distances from the Axis, \url{https://doi.org/10.1103/PhysRev.128.2384}

\bibitem{enhancements}
The Pierre Auger Collaboration (2011) The Pierre Auger Observatory V: Enhancements \url{https://arxiv.org/pdf/1107.4807.pdf}

\bibitem{firstprime}
The Pierre Auger Collaboration (2016) The Pierre Auger Observatory Upgrade - Preliminary Design Report, \url{https://arxiv.org/pdf/1604.03637.pdf} 

\bibitem{dirac}
PAM Dirac (1928) The quantum theory of the electron, Proc. R. Soc. Lond. A 1928 117 610-624, DOI: 10.1098/rspa.1928.0023, \url{http://rspa.royalsocietypublishing.org/content/117/778/610}

\bibitem{positron}
CD Anderson (1933) The Positive Electron, Phys. Rev. 43, 491, DOI: \url{https://doi.org/10.1103/PhysRev.43.491}

\bibitem{muon}
CD Anderson and SH Neddermeyer (1936) Cloud Chamber Observations of Cosmic Rays at 4300 Meters Elevation and Near Sea-Level, Phys. Rev. 50, 263, DOI: \url{https://doi.org/10.1103/PhysRev.50.263}

\bibitem{firstshowers}
P Auger, P Ehrenfest, R Maze, J Daudin, and R. A. Fr�on (1939) Extensive Cosmic-Ray Showers, Rev. Mod. Phys. 11, 288, DOI: \url{https://doi.org/10.1103/RevModPhys.11.288}

\bibitem{scifund1}
AAAS (2018) Historical Trends in Federal R\&D, \url{https://www.aaas.org/programs/r-d-budget-and-policy/historical-trends-federal-rd} retrieved on Dec. 23, 2018

\bibitem{pmthistory}
BK Lubsandorzhiev (2006) On the history of photomultiplier tube invention, NIM A 567 (2006) 236?238 DOI: 10.1016/j.nima.2006.05.221

\bibitem{colliderhistory1}
PJ Bryant (1992) A Brief history and review of accelerators,  General accelerator physics, CERN-94-01 (94/01,rec.Mar.) Conference Proceedings in Jyvaeskylae: C92-09-07.1, p.1-16

\bibitem{colliderhistory2}
G Pancheri and L Bonolis (2018) The path to high-energy electron-positron colliders: from Wider\o e's betatron to Touschek's AdA and to LEP. IOP Newsletter Jan. 2018, Retrieved Dec 23. 2018

\bibitem {ultraray}
KH Kampert and AA Watson (2012) \textit{Development of Ultra High-Energy Cosmic Ray Research} in: From Ultra Rays to Astroparticles, A Historical Introduction to Astroparticle Physics. Ed. B Falkenburg and W Rhode. Springer, available through SpringerLink. DOI: \url{https://doi.org/10.1007/978-94-007-5422-5}

\bibitem {ultraray_blau}
Michael Walter (2012) \textit{From the Discovery of Radioactivity to the First Accelerator Experiments} in: From Ultra Rays to Astroparticles, A Historical Introduction to Astroparticle Physics. Ed. B Falkenburg and W Rhode. Springer, available through SpringerLink. DOI: \url{https://doi.org/10.1007/978-94-007-5422-5}

\bibitem{haverah_lillicrap}
SC Lillicrap et al 1963 Proc. Phys. Soc. 82 95

\bibitem{haverah_watson} %contained in watson_cv.pdf
M Ave, JA Hinton, RA Vazquez, AA Watson and E Zas (2000) New Constraints from Haverah Park Data on the
Photon and Iron Fluxes of Ultra High Energy Cosmic Rays, Physical Review Letters, 85, 2244-2247

\bibitem{haverah_who} %tennent
 RM Tennent 1967 Proc. Phys. Soc. 92 622
 
 \bibitem{crapp}
 TK Gaisser (1990) Comsic Rays and Particle Physics, ISBN-13: 978-0521339315

\bibitem{kamiokande}
KS Hirata et al. (1988) Experimental Study Of The Atmospheric Neutrino Flux, Phys. Lett. B, Vol 205, num 2,3.

\bibitem{agasa}
N Chiba et al. (1992) Akeno Giant Air Shower Array (AGASA) covering 100 km$^2$ area, NIM A 311, pg. 338-349

\bibitem{casamia}
R Ong (2006) Ultra High Energy Cosmic Ray Research with CASA-MIA, Submission for Cronin Fest at University of Chicago, available at \url{http://www.astro.ucla.edu/~rene/talks/Cronin-Fest-Ong-Writeup.pdf}, retrieved Dec 27, 2018














\end{thebibliography}









\end{document}










https://phys.org/news/2013-04-schrodinger-equation.html
