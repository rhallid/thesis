% !TEX root = ../main.tex

\newpage
\begin{center}
		\begin{singlespace}
	{\huge\thetitle\unskip\strut\par}
		\end{singlespace}
	{%
	\large%
	{\expandafter\MakeUppercase\expandafter{\theauthor}}%
	\unskip\strut\par%

	}
\end{center}

\bigskip
\begin{center}
\addcontentsline{toc}{chapter}{Abstract}
	\bfseries \abstractname\vspace{-.5em}\vspace{0pt}
\end{center}
In this dissertation, we will describe work completed towards the Pierre Auger Observatory's AugerPrime Upgrade as well as auxiliary timing work, hardware design and finally a test of correlations of Starburst Galaxies with the arrival directions of Ultra High Energy Cosmic Rays (UHECRs). In the first three chapters, we review the history, observables and detection techniques of UHECR physics, both past and present. We then look at the future upgrade of Auger and give an in depth description of the firmware, software and hardware that make up the Upgraded Unified Board (UUB), which is to be at the heart of AugerPrime. A discussion of the scientific mechanisms and merits of event-by-event composition measurements is presented, and the necessity of a new board to support this is exposed. We then move into the precision timing implementation in AugerPrime, discussing GPS receiver selection and time-tagging system performance. We find that the timing resolution of the UUB is $\sigma_{det}=8.44\pm\,.15\mbox{ ns}$, and confirm it using two methods. 

Subsequent to this, we discuss auxiliary timing projects which support Auger as well as the Cherenkov Telescope Array. Results are shown for an experiment to determine spatial correlations of GPS timing errors, and hardware for timing at CTA and in the Auger@TA cross calibration is described. In the final chapter of this work, we move on to examining the recent Starburst correlation result of the Auger Collaboration, and cross check this by invoking a magnetic field model and back-tracing the arrival directions of UHECRs seen by Auger. We test to see how likely it is that the observed UHECR sky is more correlated with the observed Starburst Galaxy (SBG) sky than an isotropically chosen set of random sources. The test shows a deviation from isotropy at the 1.6$\sigma$ level. Finally, we describe future directions for SBG correlation tests.
  % \themainabstract