% !TEX root = ../main.tex

\chapter{Introduction}
\label{intro}
In the development of modern physics throughout the last century the science of cosmic rays, particles originating outside of Earth's atmosphere, has been a driving force. 
Since their first observation by Coulomb and subsequent exploration by Domenico Pacini and separately Victor Hess, the study of particles accelerated by extraterrestrial mechanisms has been a major source of physics knowledge, leading to the founding of particle physics, astrophysics, and more recently particle astronomy \cite{pacini}.  Cosmic rays are known to originate from a variety of sources, in particular low and medium energy cosmic rays are believed to originate from supernovae and their remnants, while the highest energy cosmic rays are still of unknown origin \cite{stanev}. 
% The lowest energy particles originate from the sun and interactions with solar radiation in the Earth's atmosphere. 

The study of Ultra High Energy Cosmic Rays (UHECRs) originated with experiments by Pierre Auger and his colleagues to detect extensive air showers, and was solidified by the work of John Linsley and collaborators in the experiments at Volcano Ranch \cite{linsley}. The developments in the detection mechanisms and source identification that will be discussed in this document, follow along the line of experimentation propagated by Linsley, using large arrays of detectors spread over vast distances to detect extensive air showers. Throughout the latter half of the 20th century, various techniques were used to detect UHECRs including films and electronic arrays of scintillators and water tanks. These efforts culminated in the early 2000's in the construction of the Pierre Auger Observatory. 

Initially built as an array of 1660 Water Cherenkov detectors and 24 Nitrogen Fluorescence telescopes, covering 3600 km$^2$, approximately the size of the American state of Rhode Island, the Pierre Auger Observatory has been the flagship detector for particles above an energy of $10^{18}$ eV. As time progressed, the observatory evolved, integrating many efforts to improve its ability to characterize and detect extensive air showers \cite{enhancements}. After about a decade of operation, the possibility of upgrading the observatory's capabilities across all detectors was explored, and today the collaboration is on the threshold of executing an array-wide improvement in the electronics and detection abilities of the apparatus \cite{firstprime}. This upgrade and an effort to correlate the arrival directions of UHECRs with starburst galaxies will be the main focus of this work.

\section{A brief history of Cosmic Rays and Particle Physics}
\label{history}
In the post World War One era of physics, many efforts were made to understand and utilize the advances of Quantum Mechanics and General and Special Relativity. In 1926, Erwin Schr{\"o}dinger posited his eponymous equation, describing the dynamics of wavefunctions, the fundamental description of quantum particles.  Shortly thereafter, in 1928, Paul A.~M. Dirac proposed his equation which united the principles of Quantum Mechanics and Special Relativity \cite{dirac}. With this advance, cosmic rays took center stage in the development of modern physics.%It is with this advance, that the importance of cosmic rays takes center stage.  

Integral to the Dirac equation, is the prediction of opposite-charge equal-mass particles for each known elementary particle. At the time, knowledge o the full complement of elementary particles was extremely limited, but at the very least this led to the prediction of the positron. In 1933, Carl Anderson published his paper confirming the existence of a ``positive electron". This was quickly followed, just 3 years later, by the discovery of the muon \cite{positron,muon}. In the years after this, cloud chambers with scintillator triggered cameras led to the discovery of many new particles, most of which were types of mesons.  All of these discoveries were the direct result of the study of cosmic rays.

Meanwhile, Pierre Auger and his collaborators were working to uncover particles of extremely high initial energies causing showers of lower energy particles spread over great distances \cite{firstshowers}. He and his team did this by setting up Geiger-Muller and later scintillation counters at increasing separations from each other and watching the rate at which coincidences were detected as a function of their separation. They tried this at multiple sites while monitoring external conditions such as barometric pressure. With this experiment, the existence of particles estimated to have energies upwards of 10$^{16}$ eV was confirmed. While many of the advances in particle physics at this time were driven by lower energy cosmic rays, this represents the first venture towards detecting Ultra High Energy Cosmic Rays. 
\subsubsection*{Post World War II}
After the second world war, the technological and science funding situations conspired to provide a previously-unseen level of funding for experimentation (particularly into the mid and late '50s) \cite{scifund1}. The advent of the photomultiplier tube (PMT) and the application of the first computers gave cosmic ray and particle physicists new tools \cite{pmthistory}. While versions of the photomultiplier tube existed in the 1930's, the products available in the late 40's and 50's had been refined from their predecessors. The photomultiplier tube is a device consisting of a photocathode-- a thin piece of metal off of which free electrons are generated from photons through the photoelectric effect-- multiple metal leaves and an anode collect the freed charge. A large voltage is held between the anode and the cathode such that when an electron comes off of the cathode, it is pulled towards the anode. It excites many more electrons off of the metal leaves and these electrons are similarly accelerated towards the anode by the high voltage. Through this process, a photomultiplier tube takes a photon and produces a measurable current on the order of nano or micro-amperes. %corbin says to take out the additionally but i kind of like it

\figwrap{The photo published by Carl Anderson in \cite{positron} in 1932, which led to the confirmation of the positron's existence. The cosmic rays entering his cloud chamber were put under a magnetic field, hence the bending trajectory, and pictured in the center is a lead plate, meant to stop electrons.\vspace{-30pt}}{./images/positron.png}{2 in}{l}{Positron Picture by Carl Anderson}{}

As physics advanced, the twin fields of particle physics and cosmic ray physics began to diverge. While particle accelerators had existed since the 20's (or earlier depending on how rigorously `particle acceleration' is defined), they were now becoming much more useful as fundamental probes of physics in ways that cosmic rays simply could not match up to \cite{colliderhistory1}. Fundamentally, the difference between studying particle physics using a collider versus using cosmic rays is in the predictability of the point and time of interaction. Cosmic rays, even today, provide access to much higher energy interactions, but they occur more-or-less randomly, while interactions in a collider occur in a predetermined location and time (some like to say ``in a jar"). Due to this predictability, particle physicists need to only build one detection apparatus and guarantee the interesting portions of the collision will occur inside this apparatus. Meanwhile, cosmic ray physics at this time, especially at the highest energies, began to be engineered in a distributed way, opting for dispersed detectors in the vein of Pierre Auger's experiments in the '30s. 
\figwrap{A diagram of Volcano Ranch in 1962 (\cite{volranch}): the dark circle represents a new detector being added for in depth shower timing studies.}{./images/volranch.png}{2 in}{r}{Volcano Ranch Layout}{}
Perhaps the most notable experiment\footnote{There were many noteworthy experiments that shaped the field, but we will concentrate on an outline of experiments to show the evolution of the field of UHECR physics, leading to the Auger Observatory.} pursuing this direction was the Volcano Ranch Experiment, led by John Linsley \cite{linsley}. His work at Volcano Ranch began with the setup of an array of scintillation counters, commonly referred to as scintillators (although this is a slight misnomer, see section \ref{scints}). These scintillators were placed at uniform distances apart from each other in a triangular grid. The grid spacing changed as the experiment progressed, but to give context, in 1962 it was 884m \cite{volranch}. Due to the lower speed of the electronics of his day, it was still debated whether the arrival times of the signals from each scintillation detector could be used to triangulate the arrival directions of the showers, thus increasing the grid spacing gave longer differences in arrival times and therefore made shower arrival direction reconstruction via timing a more viable approach. That said, for many of the great discoveries of Volcano Ranch, the coincidence window for detection was set such that only showers of \textless10\degree zenith would be recorded. 

It was in this configuration that Linsley's group detected the first $10^{20}$eV comic ray. Along with a handful of other data points in this regime, Linsley was able to see the flattening of the spectrum \cite{volranch}, as will be discussed in the coming sections of this chapter. The effective radius of curvature of cosmic ray showers in various experiments was also noted at this time, and it was correctly assumed by Linsley and others in the field that this curvature relates to the point of first interaction in the atmosphere, and also to the position of the maximum of particle production in the shower.

Moving forward in the evolution of cosmic ray physics, we now begin to see the development of the Air Fluorescence camera/technique. At Volcano Ranch by Linsley's group, as well as the Sydney Air Shower Array detectors called Fluorescence Telescopes (see \autoref{fluor}) were implemented. The design comes from research by the groups of Kenneth Greissen and Bruno Rossi and although it was initially unsuccessful, it ultimately resulted in the construction of the Fly's Eye detector. \cite{ultraray}. The fluorescence telescope is one of the inventions that led to a much deeper understanding of cosmic ray showers as one could now measure how the shower develops as it goes through the atmosphere. This allowed confirmation and further study of important quantities such as $X_{max}$ and the elongation rate. 

Before discussing the parallel development of emulsion film detectors, we must discuss the advancement of the water Cherenkov technique by the Haverah park experiment of the University of Leeds in England. This apparatus became one of the major High Energy Cosmic Ray detectors around the same time as the deployment of Volcano Ranch (although it only truly came into full operation about 5 years after both experiments were commissioned). The array initially consisted of 4 water Cherenkov detectors in a semi-triangular grid (as triangular as it can be with only 4 detectors) and was triggered in much the same way as the Volcano Ranch Experiment \cite{haverah_lillicrap}. Data from the Haverah Park experiment remained relevant well into the 1990's (\cite{haverah_watson}), however the lasting legacy of Haverah Park is the advancement of the water Cherenkov detection method, which previously had problems with fungus and water purity \cite{haverah_who}.

In a competing lineage to electronic detectors, emulsion film detectors were developed and deployed in the late 1930's by Marietta Blau, who deployed them high into the Alps \cite{ultraray_blau}. These films acted much like a film picture from a camera, except that they are exposed by either high energy ionizing radiation or X-rays (as in the analogous medical application). These films allowed cosmic ray researchers from the late thirties well into the eighties to record tracks left by showers over long exposure times \cite{crapp}. While these measurements are not inherently calorimetric, the precision given by being able to visually see the tracks of individual ionizing particles allows for particle counting and shower geometry methods to attempt to determine the primary energy. Film based experiments allowed researchers to directly visualize the detailed nature and progression of cosmic ray showers, allowing them to confirm and give insight into the results of other experiments which were better at collecting larger numbers of showers at higher energies.

\subsubsection*{Dawn of the Modern Era}

Looking forward from the early days of Volcano Ranch and Haverah Park, the late 1960's through the 80's opened new doors for cosmic rays physics, especially in terms of accessing new messenger particles in the form of neutrinos and very high energy photons. In this vein, new detection techniques were provided by both the Whipple Telescope, the first Imaging Atmospheric Cherenkov Telescope (IACT) and the KamiokaNDE experiment, a proton decay experiment that accidentally became the first effective neutrino telescope. The Whipple telescope is based on a Davies-Cotton optics design from the mid 1950's which uses segmented mirrors to create a large aperture that is in turn used to collect the Cherenkov light directly from air showers caused by GeV to TeV gamma rays. This proved the viability of what are now frequently referred to as gamma ray telescopes and almost all such designs are derived from the Whipple design. These include VERITAS (at the same site as the Whipple Telescope), MAGIC and HESS, as well as the upcoming CTA project \cite{ultraray}.  

On the neutrino front, the KamiokaNDE I and II experiments ran through the 80's, beginning in 1983, and while they intended to detect proton decay, their detectors were sensitive enough that they were able to detect atmospheric, solar and astrophysical neutrinos. The apparatus is a 3000 metric ton pool of water, which is ideal for a proton decay experiment as it contains a high concentration of hydrogen atoms and facilitates Cherenkov light production \cite{kamiokande}. However, the detector must be able to detect and discern signals from neutrinos and cosmic rays for background detection, and was therefore able to take data on them.
%That said, the large fiducial volume of the detector, the sides of which are lined with 1000 PMTs specially designed by Hammamtsu, makes it such that any high energy particle interacting inside will be caught (as it must for their background rejection). 
Famously, this experiment caught neutrinos from Supernova 1987A, which were likely the first ever astrophysical neutrinos measured and identified. 

In the transition from the late 1980's to the early 1990's, the field of particle astrophysics, especially UHECR astrophysics, began to resemble its current form. Through the interest of Nobel laureate James W. Cronin, and other particle physicists, as well as X-ray astronomers, the goal of figuring out the acceleration mechanisms of UHECRs came to the front of the field. Accordingly, a number of sensitive and precisely executed experiments were built; for the sake of brevity, I will highlight two: the Akeno Giant Air Shower Array (1990-2004) and CASA-MIA (1990-1997). The Akeno Giant Air Shower Array (AGASA) was a 100 km$^2$ array in Japan at the Akeno Observatory consisting of surface and buried scintillation detectors and was built with the explicit intention of observing cosmic rays of $10^{17}$ eV or higher. Some infrastructure and detectors were already installed as early as 1984 from other ventures \cite{agasa}. AGASA paved the way for the Auger Observatory, but was the direct predecessor, along with Fly's Eye, to the Telescope Array \cite{ultraray}. 

CASA-MIA, or the Chicago Air Shower Array-MIchigan Anti-coincidence muon array, set out with a slightly different objective. Instead of looking for the showers caused by hadronic primaries, the objective of CASA-MIA was to explore the arrival directions of Very High Energy (VHE) gamma rays. In the optical to low gamma ray energy regimes, only direct detection of photons is possible, usually by balloon- or space-borne detectors, but gamma rays above $\sim$10 GeV cause air showers much like UHECRs except scaled down and of much lower muon content and higher $X_{max}$. To detect these, about one thousand surface stations consisting of 4 scintillation counters (CASA) each, topped with an ancient lead (i.e. low radioactivity) sheet to encourage pair production in shower photons, and about one thousand more buried scintillation counters (MIA) were employed in a .23 km$^2$ array \cite{casamia}. 

Looking in the energy range of 10$^{14}$ eV - 10$^{16}$ eV, CASA-MIA detected gamma ray showers with its above ground scintillators, while rejecting hadronic showers by detecting their muonic component with the veto array. In terms of lineage, the CASA-MIA team provided much of the expertise and leadership for the construction of Auger. Functionally, the detector is more akin to the High Altitude Water Cherenkov (HAWC) observatory, a gamma ray detector that uses direct particle detection air shower techniques. Through the CASA-MIA experiment, much was learned about how to set up a distributed array and how to design effective detection logic. 

 %The basic idea is to catch the showers from \~ 10$^{14}$ eV - 10$^{16}$ eV gamma rays, and identify hadronic showers of similar energies through their muon multiplicity. Perhaps ironically, the CASA-MIA team provided much of the expertise and drive for the construction of Auger, but the detector itself is more of a functional predecessor to the High Altitude Water Cherenkov Experiment (HAWC), a gamma ray observatory which uses air shower techniques and direct particle detection. Through the CASA-MIA experiment, much was learned about how to set up a distributed array and how to design effective detection logic. 
 
\subsubsection*{UHECR Physics Today}

This more or less brings us to the beginning of the current era of UHECR physics, which is dominated by the Auger Collaboration (which is the primary focus of this work), and the Telescope Array (TA). A more detailed description of both experiments will be given in \autoref{auger} and \autoref{ta}, respectively, but to give context, the Pierre Auger Observatory started development in the mid 1990's and began construction in 2000 in Malargue, Mendoza Province, Argentina, at the turn of the millennium. Operation began in 2004, and has continued through to today. The TA experiment began development around the same time as Auger began operation, and was originally located at Dugway Proving grounds, in a site expanded around the Fly's Eye experiment (for the full deployment it was moved to Delta, UT). Both apparatuses measure UHECRs of 10$^{18}$ eV+ throughout the bulk of the array, and feature low energy infills for engineering tests and to expand their sensitivities to energies as low as 10$^{16}$ eV.

%you should talk about fly's eye, milagro, agasa(check), kamiokande(check), CASA-MIA(check), at least mention Haverah Park(check)
\section{Motivation and Goals}
Broadly, the work in this document aims to advance the progress towards an upgraded Pierre Auger Observatory, dubbed \textit{AugerPrime}. Auger has successfully shown that air shower arrays can cover massive areas to obtain exposures large enough to precisely fill in the highest energy portions of the cosmic ray spectrum. Its upgrade is primarily motivated by a two part case. 

First, the composition of UHECRs is a vital measurement in moving the field of charged particle astronomy forward. By understanding the composition of the highest energy cosmic rays, we can work to infer the rigidity of each particle as it propagates to the Earth. With this information, $Z$ dependent anisotropy studies may elucidate, or help elucidate, the mystery of the origin of UHECRs.

The other component of the motivation in upgrading Auger, is a modernization effort. Inexpensive and powerful electronics have developed over the past 15 years since Auger was built, and in order to ensure the continued operation, and to take advantage of increased performance, the entire electronics package is being upgraded. The scintillation and radio detectors that are to be integrated into the upgraded surface detectors will need additional data handling and processing abilities. This electronics upgrade also supports the composition measurement goal, by adding the facilities necessary to measure added channels of incoming data as well as a much more powerful onboard processing and programmable logic system. Adding a flexible programmable logic device will allow us the modify the functionality, especially the trigger logic, in-situ. 

In particular, the work in this document will concern the design of the programmable logic, the operating system software, the time-tagging logic, and the hardware and software integration of the new GPS model. Supporting these design tasks, we will also review verification data for the time-tagging system to determine the overall time resolution of the upgraded detector and show the results of testing GPS receiver models for the upgrade. Continuing along this path, results created using branches of the AugerPrime time-tagging hardware will also be included (e.g. quantification of spatial dependence of GPS timing errors, a miniaturized in situ timing system for Auger\@TA, etc.).

To round out the work included, we will report preliminary results from a follow-up to Auger's recent correlation of UHECR arrival directions with Starburst Galaxies. Towards this end, we will be using software written by the High Energy Astrophysics group at CWRU to execute and manage simulations of UHECR back propagation through the JF12 magnetic field model. We will simulate a meaningful portion of the available arrival direction data, and use a $\chi^2$-like test to infer the significance of the distribution of SBGs over an isotropic distribution.



%Moving forward in the evolution of cosmic ray physics, we now begin to see the development of the Air Fluorescence camera/technique. At Volcano Ranch by Linsley's group, as well as the Sydney Air Shower Array detectors called Fluorescence Telescopes (see \autoref{fluor}) were implemented. The design comes from the group of Kenneth Greissen, a graduate student of cosmic ray researcher Bruno Rossi, who was working on it through the mid 1960's but was somewhat unsuccessful (although his work did result in the construction of the Fly's Eye detectors) \cite{ultraray}. The fluorescence telescope is one of the inventions that led to the deep understanding of cosmic ray showers as one could now visualize how the shower develops as it goes through the atmosphere. This allowed confirmation and further study of important quantities such as $X_{max}$ and the elongation rate. 






