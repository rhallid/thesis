\chapter{Subsystems and Work Packages}
In order to divide the large amount of work that the collaboration did towards upgrading the surface detector, the effort was split into work packages. A table is provided in, with a table of participating institutions and their tasks in \autoref{wpdiag}. %In the subsections under this header, we will discuss a few of the work packages that relate to the tasks completed for this dissertation. If you are looking for details on those systems not explicitly discussed here, check \autoref{boardview}, if it is not there, then we have deemed it outside the scope of this work, for example board assembly, or array simulation.
\begin{table}[h!]
\centering
\includegraphics[width=5.5 in]{./images/wpdiag.png}
\caption[Auger Prime Institutions]{A diagram of work packages and the institutions involved in them. This is from the publicly available development plan, \bigcite{devplan}. It is worth noting that some institutions such as Fermilab (FNAL) and Ohio State (OSU) made important contributions early on, but were ultimately forced to pull out of the project due to funding restrictions. }
\label{wpdiag}

\end{table}
\begin{table}[h!]
\centering
\begin{tabular}{|c|c|} \hline
WP1 & Analog PMT signal processing development \\ \hline
WP2 & Trigger development \\ \hline
WP3 & Time tagging development \\ \hline
WP4 & Slow Control development \\ \hline
WP5 & UUB Hardware Design \& Integration \\ \hline
WP6 & UUB software development \\ \hline
WP7 & Calibration and Control tools development \\ \hline
WP8 & Assembly, Deployment and Validation \\ \hline
WP9 & Simulations and Science Validation \\ \hline
WP10 & Project Management \\ \hline
\end{tabular}
\caption[Auger Work Packages]{A table of the different work package designations. Some of these are discussed in \autoref{subsystems}.}
\label{wptab}
\end{table}
When the work packages were initially designed, the new detector type had not yet been chosen and a number of contenders were still moving forward with their research. Eventually, the decision was made to use the scintillation detector over the various other proposals, which included a small PMT to pick up saturating showers (\cite{dynrang}), splitting the tank in two for muon electron separation, and installing Resistive Plate Chambers under each station, amongst others. When the decision was made, SSD calibration and construction was added to WP1. %Minutes and official documentation are publicly available \href{https://atrium.in2p3.fr/nuxeo/nxpath/default/Atrium/sections/Public/Projet%20AUGER/Test@view_documents?tabIds=%3A&conversationId=0NXMAIN1}{here}.
